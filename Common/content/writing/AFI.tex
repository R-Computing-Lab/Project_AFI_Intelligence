\subsubsection{Generation 1 AFI}NLSY-79 subjects indicated their AFI a maximum of three times, in response to questions in 1983, 1984, and 1985. The 1984 and 1985 questions were included to assess those with non-response in 1983, but in fact many female respondents were surveyed multiple times. Further, females were asked for additional related information (Year of First Intercourse, Month of First Intercourse) in 1984 and 1985. The average AFI for women was 18.52 (sd $=$ 2.12; n $=$ 5562); men was 17.16 (sd $=$ 2.36; n $=$ 5640).

We used the repeated questions to estimate the test-retest reliability of self-reported AFI and AFI difference scores. In Table \ref{table_measurement_trt_g1afi}, the lower triangle reports the correlations of self-reported AFI across 1983-1985; the diagonal indicates the number of respondents reporting AFI for that year, and the upper triangle indicates the number of respondents that reported AFI for both respective years. The test-retest correlations are moderate to high (r $>$ .75) across all viable pairings, suggesting that our subjects are reliably reporting AFI.\medskip\\
\begin{longtable}{@{\extracolsep{5pt}}rlll} \caption{Correlation of Gen1 Self-Reported AFI Across 1983-1985}\label{table_measurement_trt_g1afi}
\partialinput{6}{12}{../Common/content/tables/table_ttafireliable_z.tex}
\end{longtable}

\subsubsection{Generation 2 AFI}Over the life-time of the NLSY-Children survey, participants were asked approximately the same questions to assess AFI that their mothers were asked. However, Generation 2 respondents were only asked for AFI information once they had reached age 15 or later. The exact nature of the question varied by administration. Between 1988 and 2000, subjects were asked for age, year, and month of first intercourse. After 2000, subjects were only asked their age.

We calculated NLSY-Children AFI using a multi-step process for three reasons: (1) to account for the diversity of AFI questions across survey administrations, (2) to incorporate multiple reports by the same participant, and (3) to account for the imprecision of AFI reporting (\eg, a subject who reports AFI at 16 could be any age between 16 years and 1 day old through 16 years and 364 days old).

For each survey, we transformed year of first intercourse into an age variable, AFI. If subjects reported both age and year within the same survey and ages were different, we averaged the AFI scores. Across surveys, we identified the earliest possible AFI and the latest possible AFI for each subject. We designated these two AFIs as the Minimum AFI and the Maximum\footnote{We added 1 year to the Maximum AFI to address the imprecision of self-reported age. The expected value of AFI of any subject does not equal the reported AFI. For example, a subject who reports AFI at 16 could be anywhere from 16 years and 1 day old to 16 years and 364 days old.} AFI respectively, thus identifying the full range of possible AFIs for each participant. We used this AFI range to calculate the expected value of AFI. Using this method, the average Generation 2 AFI was 16.01 (sd $= 2.30$; n $= 6288$).\footnote{Taking the average of all AFIs (without addressing expected value), results in 15.49 (sd = 2.30; n = 6288). Adding in expected value of .5 changes this value to 15.99, approximately our computed age.}

After transforming all AFI scores, we recoded impossible AFIs as missing. A score was impossible if the reported AFI exceeded participant's age at time of survey (new $\overline{\mathrm{AFI}} = 15.99$, sd $= 2.30$, n $= 6235$). Next, we excluded all AFIs below age 12 (new $\overline{\mathrm{AFI}} =16.14$, sd $= 2.10$, n $= 6087$). Finally we excluded subjects who reported AFI prior to menarche (new $\overline{\mathrm{AFI}} = 16.16$, sd $= 2.09$, n $= 6047$). We excluded those below age 12 because those responses likely are the result of misunderstanding, non-consensual sexual activity, or other forms of unreliability. We excluded those with pre-menarchal AFI because we were only interested in AFI that could potentially link to reproduction and fertility. AFI varied by gender and race. Most notably, women reported AFIs that were 6 months later than men, and black men reported the lowest AFI (15 yrs) of any race-gender categories. For a complete portrayal of summary statistics, see Table \ref{table_afi_race_gender} and Figure \ref{plot_afi_by_race_sex}.
\noindent\begin{minipage}{\linewidth}
\begin{longtable}{@{\extracolsep{5pt}}lllcc}
\caption{Gen2 Mean AFI by Gender, Race, and Gender by Race.}\label{table_afi_race_gender}
\partialinput{2}{18}{../Common/content/tables/table_summary_stats_AFIRACEGENDER.tex}
\end{longtable}
\end{minipage}


