First, we examined the between-family results. We tested whether the family average of Gen2 AFI could be predicted by the family averages of Gen1 intelligence and of Gen2 intelligence. We evaluated the influences both independently and simultaneously. All intelligence scores have been standardized by generation ($\overline{g} = 0$, sd $= 1$), prior to averaging by household. AFI scores have been standardized by gender ($\overline{\mathrm{AFI}} = 0$, sd $= 1$), prior to averaging by household. In Tables \ref{table_Mean_Mom_Intelligence_Mean_Child_AFI_9} - \ref{table_Mean_Joint_Intelligence_Mean_Child_AFI_9}, we have reported results for three different linking methods:
\begin{itemize} 
\item The mixed model, which contains the firstborn child of each sister;
\item The daughters model, which contains the firstborn daughters; and 
\item The sons model, which contains the firstborn sons).\end{itemize}
In the spirit of transparency, we have reported all three methods. However, because all three linking methods reported similar findings, we will focus the results section on the mixed model and only discuss the other two methods when they deviate. We have also provided zero-order and pairwise semi-partial correlations for all between-family variables (Tables \ref{table_cor} and \ref{table_spcor_btw}), using the mixed-model data and the ppcor \R library \citep{kim2015ppcor}.

\subsubsection{Gen1 Mean Intelligence $\rightarrow$ Gen2 Mean AFI} Gen1 sister averages (NLSY79 mothers) of standardized AFQT scores were used to predict Gen2 averages of gender-standardized AFI. Table \ref{table_Mean_Mom_Intelligence_Mean_Child_AFI_9} displays the results by Gen2 category. The mixed model reports the averages of the firstborn child (both males and females) of each maternal sister (n $= 342$). A one-unit increase in the average standardized intelligence of the children's mothers predicted a statistically significant increase of $.013$ standard deviations in average Gen2 AFI. When we transform the coefficient into a standardized beta weight ($\beta_{Gen1 Intell} = .299$), a standard-deviation increase in the averaged mother's intelligence predicts a .299 standard-deviation increase in the average of the cousins' AFI. The adjusted R$^{2}$ was .087.

\subsubsection{Gen2 Mean Intelligence $\rightarrow$ Gen2 Mean AFI} Gen2 averages of standardized intelligence scores were used to predict Gen2 averages of gender\hyph standardized AFI. Table \ref{table_Mean_Child_Intelligence_Mean_Child_AFI_9} displays the results by Gen2 category. The mixed model reports the averages of the firstborns of each of the NLSY79 mothers (sisters) (n $= 344$). A one-unit increase in the average standardized intelligence of the children predicted a statistically significant $\approx .075$ standard-deviation increase in average Gen2 cousins' AFI. When we transform the coefficient into a standardized beta weight ($\beta_{Gen2 Intell} = .128$), a standard-deviation increase in the averaged cousins' intelligence predicts a .128 standard-deviation increase in the average of the cousins' AFI. The adjusted R$^{2}$ was $.014$.

\subsubsection{Joint Mean Intelligence $\rightarrow$ Gen2 Mean AFI} Results from the Gen1 maternal sister averages of standardized AFQT scores and Gen2 averages of standardized intelligence scores predicting Gen2 averages of gender\hyph standardized AFI are displayed in Table \ref{table_Mean_Joint_Intelligence_Mean_Child_AFI_9}. In the mixed model, Gen1 (maternal) intelligence was significantly associated with Gen2 AFI (p $< .01$), while Gen2 (child) intelligence was not significantly associated with Gen2 AFI. A one-unit increase in the average standardized intelligence of the children's mothers predicted $.013$ standard-deviation increase in average Gen2 AFI, after controlling for Gen2 cousin averages of standardized intelligence scores. When we transform the coefficients into standardized beta weights ($\beta_{Gen1 Intell} = .303$; $\beta_{Gen2 Intell} = -.003$), a standard-deviation increase in the averaged mother's intelligence predicts a .303 standard-deviation increase in the average of the cousins' AFI. The adjusted R$^{2}$ was $.086$. The total variance explained by the Joint model ($R^{2}$ = 9.1$\%$) is nearly identical to the Gen1 model($R^{2}$=9$\%$). Gen2 intelligence explains an additional .1\% of the variance.

When we broaden our sample to all Mother-Child pairs, we see that the relationship between Gen 2 intelligence and Gen2 AFI is small $(r =.139)$, and smaller than the relationship between Gen1 intelligence and Gen2 AFI $(r=.215$; see Figure \ref{plot_gen2_afi}).

\begin{landscape}
\begin{longtable}{@{\extracolsep{5pt}}lccc} 
\caption{Between-Family: Gen1 Intelligence Predicts Gen2 AFI}\label{table_Mean_Mom_Intelligence_Mean_Child_AFI_9}
\\[-1.8ex]\hline 
\hline \\[-3.8ex] 
& \multicolumn{3}{c}{\textit{Dependent variable:} Average of Gen2 AFI} \\ 
\cline{2-4}
\partialinput{10}{21}{../Common/content/tables/table_Mean_Mom_Intelligence_Mean_Child_AFI_9.tex}\\[-7ex]
\textit{Notes:}  & \multicolumn{3}{r}{$^{*}$p$<$0.1; $^{**}$p$<$0.05; $^{***}$p$<$0.01} \\[2ex]
& \multicolumn{3}{r}{\parbox{.6\linewidth}{\footnotesize Gen1-sister averages (NLSY79 mothers) of standardized AFQT scores predict Gen2 averages of gender-standardized AFI.}} \\ 
\end{longtable}\pagebreak

\begin{longtable}{@{\extracolsep{5pt}}lccc} 
\caption{Between-Family: Gen2 Intelligence Predicts Gen2 AFI}\label{table_Mean_Child_Intelligence_Mean_Child_AFI_9}
\\[-1.8ex]\hline 
\hline \\[-3.8ex] 
& \multicolumn{3}{c}{\textit{Dependent variable:} Average of Gen2 AFI} \\ 
\cline{2-4}
\partialinput{10}{21}{../Common/content/tables/table_Mean_Child_Intelligence_Mean_Child_AFI_9.tex}\\[-7ex]
\textit{Notes:}  & \multicolumn{3}{r}{$^{*}$p$<$0.1; $^{**}$p$<$0.05; $^{***}$p$<$0.01} \\[2ex]
& \multicolumn{3}{r}{\parbox{.6\linewidth}{\footnotesize Gen2-cousin averages of standardized intelligence scores predict Gen2 averages of gender-standardized AFI.}} \\ 
\end{longtable}\pagebreak

\begin{longtable}{@{\extracolsep{5pt}}lccc} 
\caption{Between-Family: Gen1 \& Gen2 Intelligence Predict Gen2 AFI}\label{table_Mean_Joint_Intelligence_Mean_Child_AFI_9}
\\[-1.8ex]\hline 
\hline \\[-3.8ex] 
& \multicolumn{3}{c}{\textit{Dependent variable:} Average of Gen2 AFI} \\ 
\cline{2-4}
\partialinput{10}{22}{../Common/content/tables/table_Mean_Joint_Intelligence_Mean_Child_AFI_9.tex}\\[-7ex]
\textit{Notes:}  & \multicolumn{3}{r}{$^{*}$p$<$0.1; $^{**}$p$<$0.05; $^{***}$p$<$0.01} \\[2ex]
& \multicolumn{3}{r}{\parbox{.6\linewidth}{\footnotesize Gen1-sister averages (NLSY79 mothers) of standardized AFQT scores and Gen2-cousin averages of standardized intelligence scores predict Gen2 averages of gender-standardized AFI.}} \\ 
\end{longtable}
