The National Longitudinal Survey of Youth 1979 dataset(NLSY79) is based on a nationally representative household probability sample, jointly sponsored by the U.S. Bureau of Labor Statistics and the U.S. Department of Defense. On December 31, 1978, 12,686 adolescents were sampled within a household probability sample from 8,770 households. The initial sample consisted of three subsamples: \begin{itemize}\item a cross--sectional household probability sample of 6,111 non-institutionalized adolescents residing in the United States on December $31^{st}$ of 1978; \item a separate over-sampled civilian subsample of 5,295 racial minorities and disadvantaged whites; \item a representative sample of 1,280 youth serving in the U.S. Military on September $30^{th}$, 1978.\end{itemize} In the two civilian samples, subjects had birthdates that ranged from January 1, 1957 to December 31, 1964, and were between the ages of 14 and 21 on December 31, 1978; military subject's birthdates ranged from January 1, 1957 to December 31, 1961, and were between 17 and 21 years old. Participants were surveyed annually until 1994, and then surveyed biennially to the present. Two waves of planned attrition occurred. After the 1984 interview, all but 201 randomly selected members of the military sample were dropped. After the 1990 interview, all 1,643 disadvantaged whites from the oversample were dropped. Note that because there are no siblings within the military sample, it is irrelevant for the current research, as all military respondents are screened out by the requirement of having siblings within the sample. More information about the sampling process and the data can be found on the Bureau of Labor Statistics (BLS) website: \url{http://www.bls.gov/nls/nlsy79.htm}

In 1986, all biological children of the female NLSY79 participants were surveyed for the first round of the NLSY79 Children and Young Adults (NLSY-Children) survey. Now that childbearing is complete among the NLSY79 females (who were 49 to 57 in the 2014 survey), a total of 11,512 respondents are in the NLSY-Children surveys, which is continuing on a biennial basis. Participants in the NLSY79 will typically be referred to as the Generation 1 (Gen1) sample, whereas the NLSY-Children will be referred to as the Generation 2 (Gen2) sample.