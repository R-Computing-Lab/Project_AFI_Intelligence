This article presents analysis of the relationship between AFI and intelligence using two different designs: a between-family design, and a combination between- and within-family design (\ie, a within-family design includes between-family variance within it). The between-family design allowed us to replicate results obtained by previous researchers who used cross-sectional samples. The combination between- and within-family design allowed us to separately account for within- and between-family variance, to determine the source of the explanatory processes.  The logic of this separation allows us to get much closer to evaluating intelligence differences within the family to address issues of causality. The results revealed a stark contrast between the two methods, and cast doubt on the validity of past causal assertions.
\subsection{Between- vs. Within-Family Variance}
\subsubsection{Between-Family Results} Notably, the between-family analyses showed a relationship between intelligence and AFI. Thus, we were able to replicate the findings of various researchers \citep{halpern2000smart,mott1983early,Paul2000,Woodward2001}, and confirm hypotheses \ref{hyp_btw_2_2} and \ref{hyp_btw_1_2}. Our within-generation findings\footnote{We report within-generation between-family effects for intelligence correlated with AFI (\eg Gen1 intelligence correlated with Gen1 AFI)} were comparable in effect size ($d_{Gen1} =.554$; $d_{Gen2} = .281$) to Halpern \et's \citeyear{halpern2000smart} finding ($d_{Halpern} = .542$). The effects were small to medium in size. The effect for Gen1 was more similar to \citet{halpern2000smart}'s finding, likely because both findings used intelligence assessed at a later age ($\approx 16$ years), compared to the Gen2 assessment at age 9.5.

The relationship between AFI and intelligence was substantially stronger between maternal intelligence and child AFI than between the child's own intelligence and child AFI, suggesting that family-level variables rather than individual-level intelligence is the likely source of the relationship. If the child's own intelligence had been the primary causal influence on AFI we would have expected a considerably weaker cross-generational association between AFI and intelligence. Instead we find that the within-generation association is the weaker effect, suggesting that the child's own AFI is likely derivative of the child's mother's intelligence (or other between-family correlates), which are the more likely causal influence. Thus, the ``new'' and alternative interpretation of this finding would be that maternal intelligence or correlates are driving the effect, and that past between-family analyses finding a link between child's intelligence and AFI are likely because child's intelligence is indirectly measuring maternal intelligence and other between-family correlates. 

There are interpretations that would support the plausibility of this result, including maternal intelligence as a direct causal influence on their children's AFI. Smarter mothers might be more effective at encouraging their children to delay intercourse -- perhaps by effectively conveying the riskiness of sexual intercourse \citep{hutchinson2003role,mathews2009predictors}. Considering that intelligence is moderately-to-highly heritable \citep{Bouchard2004} and thus highly correlated across generations, this alternative explanation would still be consistent with the traditional between family findings, which typically do not control for maternal intelligence \citep{halpern2000smart,mott1983early,Paul2000,Woodward2001}. We note that the \citet{harden2011don} findings, using biometrically-informed data, implicated the shared environment -- a between-family source of variance -- in this causal process. Our results are entirely consistent with theirs, using a different dataset and a different methodological approach to identify important sources of variance.
 
\subsubsection{Within-Family Results} In the within-family analyses, the relationship between intelligence and AFI vanishes for both maternal intelligence and child intelligence. The child of the smarter Generation 1 mother was not more likely to delay intercourse compared to the child of the less-smart Generation 1 mother. In spite of our finding that Generation 1 intelligence was a relatively stronger predictor of Generation 2 AFI than Generation 2 intelligence, we did not find that within family differences in Generation 1 intelligence were associated with differences in Generation 2 AFI.

These results cast doubt on the alternative explanation for the between-family results we posed in the previous section. If Gen1-maternal intelligence was driving the effect as a proximal cause, we would have expected to find a significant within-family link from maternal intelligence to child AFI, which we did not. Thus, within our sample, it appears that intelligence is not the proximal cause of delayed AFI. Rather, maternal and child intelligence appear to be are indirect measures of many other between-family household features, any one of which may be more proximal as the causal explanation -- income, parental education, family interaction, \etc. Or, the whole package of these features may stand in for a general environmental factor, a ``little e,'' which indexes the quality of the home environment, which could be measured as a composite of parental income, intelligence, education, family interaction, \etc.
%
\subsection{Concluding Remarks} We interpret these results in relation to two previous findings. First, \citet{Rodgers2008AJS} used Danish twin data, and found that the link from education/cognitive ability to maternal age at first birth (AFB) was entirely accounted for by between-family variance: ``variance in AFB emerges from [IQ and education] differences between families, not differences between sisters within the same family'' \citep[][p. 202]{Rodgers2008AJS}. We have exactly the same type of result in the current study. Second, Harden and Mendle's (\citeyear{harden2011don}) results, obtained from the Add Health data, use intelligence as a predictor and AFI as an outcome, just as we did with the NLSY. Their biometrical finding of meaningful shared environmental variance is consistent with our finding of only between-family variance. Their design identifies that the source of the covariation between AFI and intelligence is in the shared environment.

Our findings cast further doubt on the direct and causal influence of intelligence. By explicitly parsing between- and within-family variance, we tested the causal link between AFI and intelligence. We employed numerous replications. We varied when we measured intelligence (Age 9.5, 10.5, 11.5; Age 16); how we measured intelligence (composite of five measures; AFQT from the ASVAB); which cohort we used (NLSYC; NLSY79); how we created kinship pairs (firstborn, first female, first male); and how related our kin-pairs were (cousins; siblings). We examined and eliminated both maternal intelligence and child intelligence as having any within-family causal etiology.

Given our findings and those of \citet{harden2011don}, we find no evidence for intelligence being a direct causal influence on AFI. Instead, we direct future researchers to look at the general family environment,``little e,'' and other between-family factors correlated with maternal intelligence as likely causes of AFI.  Although ``smart teens don't have sex (or kiss much either)'' \citep{halpern2000smart} at a descriptive level, their reasons for delaying these activities do not appear to be caused by their smartness.