This article presents the relationship between AFI and intelligence using two different designs: a between-family design, and a within-family design (that also includes between-family variance within it as well). The between-family design allowed us to replicate results obtained by previous researchers who used a cross-sectional sample. The within-family design allowed us to separate within- and between-family variance, to determine the source of the explanatory processes.  The logic of this separation allows us to get much closer to evaluating intelligence differences within the family to address issues of causality. The results revealed a stark contrast between the two methods, and cast doubt on the validity of past causal assertions.
\subsection{Between- vs. Within-Family Variance}
\subsubsection{Between-Family Results} Notably, the between-family analyses showed a relationship between intelligence and AFI. Thus, we were able to replicate the findings of various researchers \citep{halpern2000smart,mott1983early,Paul2000,Woodward2001}, and confirm hypotheses \ref{hyp_btw_2_2} and \ref{hyp_btw_1_2}. However the relationship between AFI and intelligence was substantially stronger between maternal intelligence and child AFI than between the child's own intelligence and child AFI, suggesting that family-level variables rather than individual-level intelligence is the likely source of the relationship. If the child's own intelligence had been the primary causal influence on AFI we would have expected a considerably weaker cross-generational association between AFI and intelligence. Instead we find that the within-generation association is the weaker effect, suggesting that the child's own AFI is likely derivative of the child's mother's intelligence (or other between-family correlates), which are the more likely causal influence.

Thus, the ``new'' and alternative interpretation of this finding would be that maternal intelligence or correlates are driving the effect, and that past between-family analyses finding a link between child's intelligence and AFI are likely because child's intelligence is indirectly measuring maternal intelligence/correlates. 

There are interpretations that would support the plausibility of this results, including maternal intelligence as a direct causal influence. Smarter mothers might be more effective at encouraging their children to delay intercourse -- perhaps by effectively conveying the riskiness of sexual intercourse \citep{hutchinson2003role,mathews2009predictors}. Considering that intelligence is highly heritable \citep{Bouchard2004} and thus highly correlated across generations, this alternative explaination would still be consistent with the traditional between family findings, which typically do not control for maternal intelligence \citep{halpern2000smart,mott1983early,Paul2000,Woodward2001}. However, we note that the \citet{harden2011don} findings, using biometrically-informed data, implicated the shared environment in this causal process. Our results are entirely consistent with theirs, using a different dataset and a different methodological approach to identify important sources of variance.
 
\subsubsection{Within-Family Results} In the within-family analyses, the effect vanishes for both maternal intelligence and child intelligence. The child of the smarter Generation 1 mother was not more likely to delay intercourse compared to the child of the less-smart Generation 1 mother. Moreover, in spite of our finding that the Generation 1 intelligence was a relatively stronger predictor of Generation 2 AFI, we did not find that differences in Generation 1 intelligence to be associated with differences in Generation 2 AFI. 

\subsubsection{Concluding Remarks}These results cast doubts on the alternative explanation for the between-family results we posed in the previous paragraph. For, if Generation 1 maternal intelligence was driving the effect as a proximal cause, we would have expected to find a significant within-family link from maternal intelligence to child AFI, which we did not. Rather, we think maternal and child intelligence are indirect measures of many other household features, any one of which may be more proximal as the causal explanation -– income, parental education, family interaction. Or, the whole package of these features may stand in for a general environmental factor, a ``little e,'' which indexes the quality of the home environment, which could be measured asa composite of parental income, intelligence, education, family interaction.

We interpret these results in relation to two previous findings. First, \citet{Rodgers2008AJS} used Danish twin data, and found that the link from education/cognitive ability to maternal age at first birth (AFB) was entirely accounted for by between-family variance: ``variance in AFB emerges from [IQ and education] differences between families, not differences between sisters within the same family'' \citep[][p. S202]{Rodgers2008AJS}. We have exactly the same result in the current study. Second, Harden and Mendle's (\citeyear{harden2011don}) results, obtained using the Add Health data, use intelligence as a predictor and AFI as an outcome, just as we did. Their biometrical finding of meaningful shared environmental variance is the equivalent biometrical result to our finding of only between-family variance.  But our design allows us to cast further doubt on the direct and causal influence of maternal intelligence, and leaves the general family environment and other factors correlated with maternal intelligence as the likely causal influences on AFI.