To conduct our study using the requisite within-family information, we require sister pairs in Generation 1 who both had children. The children of these sisters are cousin pairs. In the original NLSY79 and NLSY-Children surveys, identification of level of sibling relatedness in the NLSY was primarily inferential. NLSY79 twins, full siblings, half siblings, and adoptive siblings were distinguishable indirectly from respondent and maternal information about birthdates and the biological father(s). NLSY-Children respondents within a given family were all full- or half-siblings, because they were (by design) the biological children of the NLSY79 females. In 2006, both NLSY surveys included explicit indicators of the level of sibling relatedness. Our research team has recently completed a multi-year project to reliably and validly identify the kinship pairs within these two datasets \citep{nlsylinksbgpaper}. Sibling and cousin pairs from these kinship links are used in the current study.

Specifically, Mother-Child-Aunt-Nibling (MCAN) tetrads were created using the NLSY Kinship Links \citep{nlsylinksbgpaper} and supporting \R package \citep{nlsylinksr}. The oldest two female kin (Mother, Aunt) were selected from each NLSY79 household (note that additional female Generation 1 sister pairs from families with three or more sisters -- a relatively small number -- were excluded). Three tetrad designs were employed, in which the genders of Generation 2 were the defining feature: 
\begin{itemize}\item Mother-Daughter-Aunt-Niece (MDAN) tetrads included the oldest Generation 2 female child from each of the two Generation 1 sisters, 
\item Mother-Son-Aunt-Nephew (MSAN) tetrads included the oldest Generation 2 male child from each of the Generation 1 sisters, and 
\item The first two types of tetrads were combined together into Mother-Child-Aunt-Nibling (MCAN) included the firstborn child from each of the Generation 1 sisters. (Note: ``Nibling'' refers to a niece or nephew with unspecified gender; compare to ``Sibling.'')\end{itemize} 
