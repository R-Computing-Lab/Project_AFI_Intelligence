Teenage sexual activity has interested academics in demography and psychology for many years \citep{Brooks-Gunn1989,kinsey1948sexual,santelli2000adolescent}. Anecdotal evidence from the popular media, (\eg, MTV's reality television franchise, \textit{16 and Pregnant}\nocite{mtv}), 
and academic research converge to support a relatively consistent secular decline in age at first intercourse  \citep[AFI; see][]{bozon2003,finer2007trends,Kann2014}. Early AFI is associated with downstream consequences, including lower educational attainment \citep{Harden2012,Spriggs2008,Wellings2001}, failure to meet education and career goals \citep{halpern2000smart}, increased risk of teenage pregnancy \citep{Leitenberg2000,Wellings2001}, and increased rates of sexually transmitted infections \citep[STIs;][]{kaestle2005young}. Moreover, beyond the obvious benefit of avoiding those negative outcomes, delaying AFI is associated with greater relationship satisfaction, perception of increased attractiveness, and higher household income \citep{Harden2012}. Because many of the negative consequences above are severe and long-reaching, it is important to identify the causal mechanisms associated with early AFI. One potential factor exerting causal influence on AFI is intelligence.

Higher levels of intelligence are associated with delaying first intercourse \citep{halpern2000smart,mott1983early,Paul2000,Woodward2001}, and also with delay of less-intimate sexual involvement \citep{halpern2000smart}. Specifically, intelligent individuals may delay intercourse to ``safeguard'' their futures \citep{kirby2002effective, manlove1998influence, raffaelli2003sexual}. They perceive the risks associated with early intercourse, (\eg, pregnancy, STIs) to have life- and career-shattering outcomes \citep{halpern2000smart,harden2011don}. Although the link between intelligence and AFI has face validity, and has been confidently asserted (or often implied) as a causal link, a fundamental confound exists in most past research that limits our ability to infer causality.

Virtually all of the AFI-intelligence literature has used between-family designs. In analysis of data from such designs, a number of genetic and environmental influences, such as education and maternal intelligence are confounded \citep{DOnofrio2013,harden2014genetic,Lahey2010,Rodgers2000}. By ignoring such confounds, the source of variance is ambiguous, and researchers that attribute the source to specific between- or within-family sources risk misattributions of causality \citep{Rowe1997,Rutter2007}. There is the potential for this type of confound in virtually all past research on the link between intelligence and AFI \citep{harden2011don,harden2014genetic,plomin2004intelligence,rodgers1999nature,rodgers1994df}. Thus, we need to critically evaluate whether intelligence has a causal influence on AFI or is rather a theoretically attractive confound. To resolve some of these methodological challenges, we use design innovations that emerge from the excellent cross-generational and longitudinal structure of the National Longitudinal Survey of Youth (NLSY; we use both the original NLSY79 survey and the NLSY-Children survey, described below).
%
\section{Cause or Confound?}
There are numerous theories that address the motivations for adolescents' initiation of first intercourse (see \citealp{Rodgers1996} or \citealp{Buhi2007} for reviews), and even more specific precursors to first intercourse \citep{Buhi2007,DOnofrio2010,kirby2002antecedents,miller1997timing,santelli1992risk}. Many of these theories emphasize biology/genetics, as adolescent pubertal development (and associated hormone changes) drives the onset of sexual behavior \citep{miller1999dopamine,udry1979age,udry1994nature}. Other theoretical frameworks use social/environmental processes to explain developing sexual involvement in adolescence, such as Social Learning \citep{diblasio1990adolescent,hogben1998using}, where social norms affect the likelihood of early sexual behavior; or Social Control theory \citep{hirschi2002causes}, where societal and cultural influences reduce the likelihood that individuals will act on their natural tendency toward sexual involvement. Under these environmental theories the underlying biology is typically ignored (\eg, Social Control theory), whereas under many of the biological/genetic theories, the environmental components are often ignored.

However, numerous articles have also advocated integrative models \citep[See][]{harden2008rethinking,harden2014genetic,udry1995sociology}. The integrative Biopsychosocial Model acknowledges both genetic and environmental contributions to human behavior \citep{Engel1977,petersen1987nature,rodgers1999nature}. Indeed, biology, psychology, and society/culture jointly influence adolescents' decisions to engage in sexual intercourse \citep{Meschke2000,zimmer2008ten}.
%
\subsection{Intelligence as a Cause}
The short-term risks of early AFI are primarily negative, whereas the rewards for delay are primarily positive. These consequences extend into adulthood -- early AFI has been related to adult delinquency \citep{harden2008rethinking}, anti-social behavior, and substance abuse \citep{boislard2011individual}, whereas those with delayed AFI have higher household incomes in adulthood \citep{Harden2012}. It is intuitively appealing to believe that intelligent individuals are more likely to observe this potential risk-reward trade off, and through volition act upon such observations by delaying first intercourse. Accordingly, intelligent individuals perceive the consequences of early AFI to negatively influence their careers \citep{halpern2000smart,harden2011don}.

Indeed, most of the literature has contributed to the expectation that intelligence is causally connected to AFI. Those with higher educational goals delay their first intercourse \citep{boislard2011individual,schvaneveldt2001academic}, whereas those who engaged in early sexual intercourse reduced their educational goals compared to earlier higher goals \citep{schvaneveldt2001academic}. Beyond academic goals, those with a greater affinity for risk and those who perceive benefits from teen-pregnancy are more likely to engage in risky sexual activities \citep{raffaelli2003sexual}. A greater understanding of the risks associated with sexual intercourse, such as HIV transmission, is also associated with delayed AFI \citep{mathews2009predictors}.

Smarter adolescents are more likely to report delayed intercourse \citep{halpern2000smart,mott1983early,Paul2000,Woodward2001}. Besides delaying first intercourse, smarter individuals appear to postpone all sexual/romantic activity \citep{halpern2000smart}. Such blanket delays may be a proactive attempt to avoid ``gateway'' activities that might lead to intercourse. Thus, many researchers have concluded that ``[h]igher intelligence operates as a protective factor against early sexual activity during adolescence, and lower intelligence, to a point, is a risk factor.'' \citep[][p. 213]{halpern2000smart}.

However, \citet{halpern2000smart} and many of the other studies we have referenced \citep[e.g.,][]{mathews2009predictors,miller1997timing,Paul2000} used between-family, typically cross-sectional, designs. Such designs cannot logically distinguish between processes that act to create differences between families and processes that create differences among family members \citep{Lahey2010,Rodgers2000}. Thus the previous studies do no provide conclusive evidence that intelligence is the causal influence behind the AFI-intelligence relationship. Logically, other alternatives are that AFI has a causal link to intelligence (which is unlikely, for the obvious theoretical reasons, including that a child's intelligence precedes AFI in time) or that other confounds cause these two outcomes to correlate, but not causally. There are dozens, perhaps hundreds, of such confounds that can logically contend to explain the link between child intelligence and AFI; we review those confounds in the next section.
%
\subsection{Intelligence as a Confound}
An equally valid set of explanations exist in which intelligence is not the causal factor behind the AFI-intelligence relationship, but rather one of dozens of correlated potentially explanatory processes. Instead, various confounds including family-level selection effects, or third variables at the individual or family level could be causing the relationship. Between-family influences such as SES and maternal intelligence could drive the relationship. Socioeconomic status is associated with the onset of first intercourse \citep{Lammers2000}, explains many of the negative consequences linked with teenage pregnacy \citep{geronimus1992socioeconomic, and is correlated with intelligence \citep{murray1998income,Neisser1996,Strenze2007}. Parental intelligence and parental education are also linked with child intelligence \citep{Bouchard2004,devlin1997heritability,mercy1982familial}, and pose viable alternative explanations in which parents could be influencing, or even actively  dissuading, their children from engaging in early intercourse. For example, daughters whose mothers communicated frequently about the risk associated with sexual intercourse were less likely to have unprotected sex and engaged in sex less frequently \citep{hutchinson2003role}. Thus it could be that intelligent mothers, not intelligent children, are the ones recognizing the consequences of early intercourse and acting accordingly. In order to better understand the causal relation between intelligence and AFI, we need to be able to untangle between- and within-family processes, using both data and designs that have the ability to separate these sources of variance.

Indeed many such findings that link intelligence with various outcomes are quite possibly the result of misattributing between-family confounds to individual-level and within-family causes. The relationship between birth order and intelligence is a classic example of this misattribution \citep[See][]{damian2015associations,Rodgers2000,rodgers2014birth}. We briefly review that research arena here, because it illustrates the same challenge that occurs in studying the link between intelligence and AFI. 

Between-family studies that rely on cross-sectional data have consistently found that first born children have higher IQs than later born children \citep{belmont1973birth,zajonc1976family}. Yet within-family studies have typically found a non-significant relationship (\citealp{berbaum1980intellectual,galbraith1982sibling,retherford1991birth,Rodgers2000}; also see \citealp{barclay2015within} and \citealp{bjerkedal2007intelligence} for recent exceptions that have found small, but significant within-family effects in large national studies). Moreover, when designs that can distinguish within- and between-family variance have been conducted, the methodological source of the IQ-birth order effects have emerged from the between-family variance \citep{black2011older,rodgers1984confluence,Rodgers2000,Wichman2006,Wichman2007}. Potential causes of this confound include maternal age at first birth, parental IQ, parental education, and SES \citep[][also see \citep{Anastasi1956} for an insightful overview, written prior to the IQ-birth order debate.]{page1979family,Rodgers2001admixture,Rodgers2008AJS}.

In summary, if there is a valid within-family link between intelligence and birth order (which is, definitionally, a within-family variable), it is at one extreme of small magnitude and only detectable in large national datasets, and at the other extreme non-existent (even in large U.S. datasets). Similarly, the link between intelligence and AFI may be mostly or completely spurious. For example, one study found that socioeconomic status is associated with the onset of first intercourse \citep{Lammers2000}, which implicates a between-family process as explanatory of much of the negative consequence early AFI and teenage pregnancy.

\subsection{Prior Within-Family Analyses} Two past studies have explicitly separated between- and within-family influences on the AFI-intelligence relationship \citep{harden2011don,nedelec2012exploring}. Harden and Mendle used 536 same-sex twin pairs from the Add Health Study to ``test[ ] whether relations between intelligence, academic achievement and age at first sex were due to unmeasured genetic and environmental differences between families.'' Twins who differed in their intelligence or their academic achievement did not differ in their age at first intercourse. They concluded that ''the association between intelligence and age at first sex could be attributed entirely to \textit{unmeasured environmental differences between families}.''(italics added, our own emphasis). Nedelec, Schwartz, Connolly, and Beaver (\citeyear{nedelec2012exploring}) conducted an exploratory analysis of MZ twin pairs from the same sample used by Harden and Mendle, using intelligence difference scores to predict various social outcomes. They found consistent null results, though their samples were small and their statistical analyses were substantially underpowered.
