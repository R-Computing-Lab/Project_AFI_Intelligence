To summarize, the current study examines the relationship between intelligence and age at first intercourse, using maternal siblings and their children from a multi-generational nationally representative sample, the NLSY. This examination extends the intelligence literature in several key ways. First, we test whether the relationship between intelligence and AFI existed in either or both between- and within-family analyses, using data in which we can explicitly separate those sources of variance. Second, we evaluated the alternative explanation that maternal intelligence influences child AFI, using the cross-generational structure of the NLSY. Third, we replicated our findings using two different age periods. Fourth, we address overlapping questions to those studied in Halpern \et (\citeyear{halpern2000smart}, Harden and Mendle (\citeyear{harden2011don}), and Nedelec \et (\citeyear{nedelec2012exploring}), using a different data source.\footnote{All of those previous studies used the Add Health dataset.}

We made the following predictions, based primarily upon \citet{harden2011don}:\\ 
Between Families,\\
1.\hypothesis{btw_2_2} Does Generation 2 (\ie, children's) intelligence predict Generation 2 AFI?: We expect intelligence to be associated with age of first intercourse because there is a sizable body of literature reporting that result \citep{kirby2002effective, manlove1998influence, raffaelli2003sexual, rodgers1994df}.\\ 
2.\hypothesis{btw_1_2} Does Generation 1 (\ie, mother's) intelligence predict Generation 2 AFI?: We also expect maternal intelligence to be associated with age of first intercourse because the heritability of intelligence is moderate-to-high \citep{Bouchard2004,devlin1997heritability}. If intelligence does causally influence AFI we would expect that the cross-generational association between AFI and intelligence would be weaker, but existent. However, if the intelligence-AFI relationship is the product of between-family confounds, then we would expect that the cross-generational association between AFI and intelligence would be stronger than the within-generation association because maternal intelligence would be more closely linked with household SES and various parental causes. In other words, maternal intelligence can serve as a proxy for many of the between-family confounds that are of concern in the current study. Comparably sized effects would also be consistent with a between-family confound. Given that \citet{harden2011don} found no within-family effect for intelligence, we expect that maternal intelligence will have a comparable or larger effect on between-family AFI than child intelligence.\\
Within Families,\\
3.\hypothesis{wth_2_2} Does Generation 2 intelligence predict Generation 2 AFI?: We do not expect to find within-family differences in intelligence correlating with AFI, given that \citet{harden2011don} did not report an effect.\\
4.\hypothesis{wth_1_2} Does Generation 1 intelligence predict Generation 2 AFI?: Unknown: it is possible that maternal intelligence will have an effect, as such a link would explain the between-family effects as well as many of the alternative household-level influences.\\
5.\hypothesis{btw_v_wth} Is the relationship consistent across cross-sectional and within-family designs?: Doubtful, we do not expect the results to be consistent across methods because both \citet{harden2011don} and \citet{Meredith2013} found no within-family effect, whereas the traditional findings from between-family studies find an effect \citep{kirby2002effective, manlove1998influence, raffaelli2003sexual}. The birth order literature is one example in which apparent causal links mitigate or disappear when variance is properly attributed to within- versus between-family sources.