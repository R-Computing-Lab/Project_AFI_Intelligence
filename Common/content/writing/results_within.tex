We replicated the between-family analyses reported in the previous subsection, adding within-family difference scores as independent variables to the between-family means. Using the discordant sibling model, we predicted the differences in Generation 2 AFI as a function of differences in intelligence, controlling for means of the outcomes and predictors. We ran three series of models, where we examined the individual and then joint influence of Gen1 intelligence and Gen2 intelligence. Moreover, within each series we included three Generation 2 linking method variants, just as we did in the between family analyses: the mixed model reports the differences of the firstborns of each sister, the daughters model reports the differences of the firstborn girls, and the sons model reports the differences of the firstborn sons. In the spirit of transparency, we have reported all three methods. However, because all three linking methods resulted in similar findings, we will focus the results section on the mixed model and only discuss the other two methods when they deviate.

\subsubsection{Gen1 Intelligence Differences $\rightarrow$ Gen2 AFI Differences} 
Generation 1 maternal sister differences in standardized AFQT scores were used to predict Generation 2 (cousin) differences of gender-standardized AFI, controlling for Generation 1 sister averages of standardized AFQT scores and Generation 2 averages of gender-standardized AFI. Table \ref{table_Dif_Mom_Intelligence_Dif_Child_AFI_9} displays the results by Generation 2 linking method. The mixed model reports the averages and differences of the firstborns of each sister (n $= 336$). Generation 2 averages of gender-standardized AFI (between-family measures) were significant predictors of Generation 2 differences in gender-standardized AFI (p $< .01$). A one-unit increase in the average gender-standardized AFI predicted $0.303$ standard-deviation increase in average Gen2 AFI difference, controlling for all other variables in the model. When we transform the coefficients into standardized beta weights ($\beta_{Gen2 Mean AFI} = .283$; $\beta_{Gen1 Mean Intell} = -.057$; $\beta_{Gen1 Diff Intell} = .03$), a standard-deviation increase in the averaged cousin's AFI predicts a .283 standard-deviation increase in the difference between the cousins' AFI. The adjusted R$^{2}$ was .066. The Gen1 intelligence difference variable was not statistically significant.

In the sons model, the Generation 1 maternal sister average of standardized AFQT scores was a significant predictor of differences in Gen2 AFI (p $< .01$). A one-unit increase in the average standardized intelligence of the children's mothers predicted $.0083$ decrease in the AFI difference between cousins. When we transform the coefficients into standardized beta weights ($\beta_{Gen2 Mean AFI} = .353$; $\beta_{Gen1 Mean Intell} = -.174$; $\beta_{Gen1 Diff Intell} = .006$), a standard-deviation increase in the averaged cousin's AFI predicts a .353 standard-deviation increase in the difference between the cousins' AFI, while a standard-deviation increase in the averaged mother's AFQT predicts a .174 standard-deviation decrease in the difference between the cousins' AFI. All other variables were not significant, including all kin-difference variables (the within-family measures). Note that this between-family result is somewhat anomalous, because it is in the opposite direction to the other results. The adjusted R$^{2}$ was $= .106$.

In the sons model for the age 10.5 and 11.5 replications (see Appendices \ref{appen10} and \ref{appen11}), the Generation 1 cousin average of standardized intelligence scores was a significant predictor of differences in Generation 2 AFI (p $< .05$; see Tables \ref{table_Dif_Mom_Intelligence_Dif_Child_AFI_10} \& \ref{table_Dif_Mom_Intelligence_Dif_Child_AFI_11}). The adjusted R$^{2}$s were similar. 

\subsubsection{Gen2 Intelligence Differences $\rightarrow$ Gen2 AFI Differences}
Gen2 cousin differences in standardized intelligence scores were used to predict Gen2 differences of gender-standardized AFI, controlling for Gen2 cousin averages of standardized intelligence scores and gender-standardized AFI (to account for between-family variance). Table \ref{table_Dif_Child_Intelligence_Dif_Child_AFI_9} displays the results by Generation 2 categories. The mixed model reports the averages and differences of the firstborns of each sister (n $= 291$). Generation 2 averages of gender-standardized AFI were significant predictors of Generation 2 differences in gender-standardized AFI (p $< .01$). A one-unit increase in the average gender-standardized AFI predicted $0.357$ standard-deviation increase in average Gen2 AFI difference, controlling for all other variables in the model. The adjusted R$^{2}$ was $= .103$. When we transform the coefficients into standardized beta weights ($\beta_{Gen2 Mean AFI} = .337$; $\beta_{Gen2 Mean Intell} = -.091$; $\beta_{Gen2 Diff Intell} = .068$), a standard-deviation increase in the averaged cousin's AFI predicts a .337 standard-deviation increase in the difference between the cousins' AFI. The Gen2 intelligence difference variable was not statistically significant.

In the sons model, the Generation 2 cousin average of standardized intelligence scores was a significant predictor of differences in Generation 2 AFI (p $< .05$). A one-unit increase in the average standardized intelligence of the children predicted a $.107$ decrease in the AFI difference between cousins(again, the sons model result is in the opposite direction to other results). When we transform the coefficients into standardized beta weights ($\beta_{Gen2 Mean AFI} = .372$; $\beta_{Gen2 Mean Intell} = -.15$; $\beta_{Gen2 Diff Intell} = .021$), a standard-deviation increase in the averaged cousin's AFI predicts a .372 standard-deviation increase in the difference between the cousins' AFI, whereas a standard-deviation increase in the averaged cousin's intelligence predicts a .15 standard-deviation decrease in the difference between the cousins' AFI. All other variables were not significant, including all kin-difference variables. The adjusted R$^{2}$ was $.132$). 

In the mixed, daughters, and sons models for the age 10.5 and 11.5 replications (see Appendices \ref{appen10} and \ref{appen11}), the Generation 2 cousin average of standardized intelligence scores were significant predictors of differences in Generation 2 AFI (p $< .05$; see Tables \ref{table_Dif_Child_Intelligence_Dif_Child_AFI_10} \& \ref{table_Dif_Child_Intelligence_Dif_Child_AFI_11}). Regardless, all kin-difference variables were not significant. The adjusted R$^{2}$s were similar. 

Moreover, when we replicate this analysis using the larger Gen1 sample of siblings, we find that the between-family relationship of Gen1 AFI and Gen1 intelligence is moderate $(r =.291)$, but within the family, the AFI of the smarter sibling was no later than the less smart sibling's. In Figure \ref{plot_gen1_afi}, we have illustrated this finding using a scatter plot for the individual subjects, and two marginal distributions to show the mean levels of AFQT and AFI, grouped by sibling classification. The distribution above the x-axis shows the distributions of AFQT based on sibling classification. The blue distribution shows the standardized AFQT score for the siblings who were .33 standard-deviations smarter than their sibling, the red distribution shows the standardized AFQT score of siblings who were at least .33 standard deviations less smart than their sibling, and the purple distribution shows those siblings who were within at least .33 standard deviations of one another. As expected, the blue distribution of smarter siblings had a higher mean than the other two groups. The distribution across from the y-axis shows the distribution of AFI based on those same sibling classifications. All three of these distributions are indistinguishable, which suggests that siblings do not differ in their AFI across these intelligence categories. This plot illustrate that there is no within-family effect for intelligence on AFI. Had there been a within-family effect of intelligence on AFI, the marginal distributions of AFI would be visually separable, and would follow the pattern or the separate distributions of AFQT.%\pagebreak

\subsubsection{Joint Intelligence Differences $\rightarrow$ Gen2 AFI Differences}
Gen1 maternal sister differences in standardized AFQT scores and Gen2 cousin differences in standardized intelligence scores were used simultaneously to predict Gen2 differences of gender-standardized AFI, controlling for Gen1-sister averages of standardized AFQT scores, Gen2-cousin averages of standardized intelligence scores, and Gen2-cousin averages of gender-standardized AFI. Table \ref{table_Dif_Joint_Intelligence_Dif_Child_AFI_9} displays the results by Generation 2 categories. The mixed model reports the averages and differences of the firstborns of each sister (n $= 285$). Gen2 averages of gender-standardized AFI were significant predictors of Generation 2 differences in gender-standardized AFI (p $< .01$), across all three linking methods. A one-unit increase in the average gender-standardized AFI predicted $\approx 0.38$ standard-deviation increase in Generation 2 AFI difference, controlling for all other variables in the model. All other variables were not significant, including all kin difference variables. When we transform the coefficients into standardized beta weights ($\beta_{Gen2 Mean AFI} = .324$; $\beta_{Gen2 Mean Intell} = -.091$; $\beta_{Gen1 Mean Intell} = .017$; $\beta_{Gen2 Diff Intell} = .059$; $\beta_{Gen1 Diff Intell} = .02$), a standard-deviation increase in the averaged cousin's AFI predicts a .345 standard-deviation increase in the difference between the cousins' AFI. The adjusted R$^{2}$ was $.090$.\\
\begin{landscape}
\begin{longtable}{@{\extracolsep{5pt}}lccc} 
\caption{Within-Family: Gen1 Differences in Intelligence Predict Gen2 Differences in AFI}\label{table_Dif_Mom_Intelligence_Dif_Child_AFI_9}
\\[-1.8ex]\hline 
\hline \\[-3.8ex] 
& \multicolumn{3}{c}{\textit{Dependent variable:} Differences in Gen2 AFI} \\ 
\cline{2-4}
\partialinput{10}{22}{../Common/content/tables/table_Dif_Mom_Intelligence_Dif_Child_AFI_9.tex}\\[-5ex]
\textit{Notes:}  & \multicolumn{3}{r}{$^{*}$p$<$0.1; $^{**}$p$<$0.05; $^{***}$p$<$0.01} \\[2ex]
& \multicolumn{3}{r}{\parbox{.6\linewidth}{\footnotesize Gen1-sister differences in standardized AFQT scores predict Gen2 differences in gender-standardized AFI.}} \\ 
\end{longtable}\pagebreak

\begin{longtable}{@{\extracolsep{5pt}}lccc} 
\caption{Within-Family: Gen2 Differences in Intelligence Predict Gen2 Differences in AFI}\label{table_Dif_Child_Intelligence_Dif_Child_AFI_9}
\\[-1.8ex]\hline 
\hline \\[-3.8ex] 
& \multicolumn{3}{c}{\textit{Dependent variable:} Differences in Gen2 AFI} \\ 
\cline{2-4}
\partialinput{10}{23}{../Common/content/tables/table_Dif_Child_Intelligence_Dif_Child_AFI_9.tex}\\[-6ex]
\textit{Notes:}  & \multicolumn{3}{r}{$^{*}$p$<$0.1; $^{**}$p$<$0.05; $^{***}$p$<$0.01} \\[2ex]
& \multicolumn{3}{r}{\parbox{.6\linewidth}{\footnotesize Gen2-cousin differences in standardized intelligence scores predict Gen2 differences in gender-standardized AFI.}} \\ 
\end{longtable}\end{landscape}
\pagebreak
\begin{landscape}
\begin{longtable}{@{\extracolsep{5pt}}lccc}
\caption{Within-Family: Gen1 \& Gen2 Differences in Intelligence Predict Gen2 Differences in AFI}\label{table_Dif_Joint_Intelligence_Dif_Child_AFI_9}
\\[-1.8ex]\hline 
\hline \\[-3.8ex] 
& \multicolumn{3}{c}{\textit{Dependent variable:} Differences in Gen2 AFI} \\ 
\cline{2-4}
\partialinput{10}{25}{../Common/content/tables/table_Dif_Joint_Intelligence_Dif_Child_AFI_9.tex}\\[-8ex]
\textit{Notes:}  & \multicolumn{3}{r}{$^{*}$p$<$0.1; $^{**}$p$<$0.05; $^{***}$p$<$0.01} \\[2ex]
& \multicolumn{3}{r}{\parbox{.6\linewidth}{\footnotesize Gen1-sister differences in standardized AFQT scores and Gen2-cousin differences in standardized intelligence scores predict Gen2 differences in gender-standardized AFI.}} \\ 
\end{longtable}
\end{landscape}