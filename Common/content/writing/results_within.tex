We replicated the between-family analyses reported in the previous subsection, using within-family difference scores and means. Using the discordant sibling model, we predicted the differences in Generation 2 AFI as a function of differences in intelligence, controlling for means of the outcomes and predictors. We ran three series of models, where we examined the individual and then joint influence of Gen1 intelligence and Gen2 intelligence. Moreover, within each series we included three Generation 2 linking method variants, just as we did in the between family analyses: the Mixed model reports the differences of the first borns of each sister, the Daughters model reports the differences of the first born girls, and the Sons model reports the differences of the first born sons. 

\subsubsection{Gen1 Intelligence Differences $\rightarrow$ Gen2 AFI Differences} 
Generation 1 sister differences in standardized AFQT scores were used to predict Generation 2 differences of gender standardized AFI, controlling for Generation 1 sister averages of standardized AFQT scores and Gen2 averages of gender standardized AFI. Table \ref{table_Dif_Mom_Intelligence_Dif_Child_AFI_9} displays the results by Generation 2 linking method. The Mixed model reports the averages and differences of the first borns of each sister (n $= 336$), the Daughters model reports the averages and differences of the first born girls (n $= 258$), and the Sons model reports the averages and differences of the first born sons (n $= 278$). All three models reveal similar results. Generation 2 averages of gender standardized AFI (between-family measures) were significant predictors of Gen2 differences in gender standardized AFI (p $< .01$), across all three linking methods. A one unit increase in the average gender standardized AFI predicted $\approx 0.34$ standard deviation increase in average Gen2 AFI difference, controlling for all over variables in the model. 

In the Sons model, the Generation 1 sister average of standardized AFQT scores was a significant predictor of differences in Gen2 AFI (p $< .01$). A one unit increase in the average standardized intelligence of the children's mothers predicted $\approx .0083$ decrease in the AFI difference between siblings. All other variables were not significant, including all kin difference variables (the within-family measures). Note that this result is somewhat anomalous, because it is in the opposite direction to the other results. The adjusted R$^{2}$ varied slightly by Generation 2 linking method (Mixed $= .066$, Daughters $= .072$, Sons $= .106$).

\subsubsection{Gen2 Intelligence Differences $\rightarrow$ Gen2 AFI Differences}
Gen2 cousin differences in standardized intellectual ability scores were used to predict Gen2 differences of gender standardized AFI, controlling for Gen2 cousin averages of standardized ability scores and gender standardized AFI (to account for between-family variance). Table \ref{table_Dif_Kid_Intelligence_Dif_Child_AFI_9} displays the results by Generation 2 categories. The Mixed model reports the averages and differences of the first borns of each sister (n $= 291$), the Daughters model reports the averages and differences of the first born girls (n $= 223$), and the Sons model reports the averages and differences of the first born sons (n $= 238$). All three models reveal similar results. Gen2 averages of gender standardized AFI were significant predictors of Generation 2 differences in gender standardized AFI (p $< .01$), across all three linking methods. A one unit increase in the average gender standardized AFI predicted $\approx 0.38$ standard deviation increase in average Gen2 AFI difference, controlling for all over variables in the model. 

In the Sons model, the Generation 2 cousin average of standardized ability scores was a significant predictor of differences in Generation 2 AFI (p $< .05$). A one unit increase in the average standardized intelligence of the children predicted $\approx .107$ decrease in the AFI difference between siblings (again, the Sons model result is in the opposite direction to other results). All other variables were not significant, including all kin difference variables. The adjusted R$^{2}$ varied slightly by Generation 2 linking method (Mixed $= .103$, Daughters $= .121$, Sons $= .132$).

Moreover, when we replicate this analysis using the larger Gen1 sample of siblings, we find that the between-family relationship of Gen1 AFI and Gen1 general ability is moderate $(r =.291)$, but that within the family, the smarter sibling (as illustrated by the blue marginal distributions in Figure \ref{plot_gen1_afi}) were indistinguishable from the less smart sibling (the red marginal distributions). Had there been a within-family effect of intelligence on AFI, the marginal distributions of AFI would not overlap.\pagebreak

\subsubsection{Joint Intelligence Differences $\rightarrow$ Gen2 AFI Differences}
Generation 1 sister differences in standardized AFQT scores and Gen2 cousin differences in standardized intellectual ability scores were used simultaneously to predict Generation 2 differences of gender standardized AFI, controlling for Generation 1 sister averages of standardized AFQT scores, Gen2 cousin averages of standardized ability scores, and Gen2 cousin averages of gender standardized AFI. Table \ref{table_Dif_Joint_Intelligence_Dif_Child_AFI_9} displays the results by Generation 2 categories. The Mixed model reports the averages and differences of the first borns of each sister (n $= 285$), the Daughters model reports the averages and differences of the first born girls (n $= 217$), and the Sons model reports the averages and differences of the first born sons (n $= 235$). All three models reveal similar results. Gen2 averages of gender standardized AFI were significant predictors of Generation 2 differences in gender standardized AFI (p $< .01$), across all three linking methods. A one unit increase in the average gender standardized AFI predicted $\approx 0.38$ standard deviation increase in Generation 2 AFI difference, controlling for all over variables in the model. 

All other variables were not significant, including all kin difference variables. The adjusted R$^{2}$ varied slightly by Generation 2 linking method (Mixed $= .090$, Daughters $= .105$, Sons $= .131$).\\
\begin{landscape}
\begin{longtable}{@{\extracolsep{5pt}}lccc} 
\caption{Gen1 Intelligence Differences $\rightarrow$ Gen2 AFI Differences}\label{table_Dif_Mom_Intelligence_Dif_Child_AFI_9}
\\[-1.8ex]\hline 
\hline \\[-1.8ex] 
& \multicolumn{3}{c}{\textit{Dependent variable:} Differences in Gen2 Mean AFI} \\ 
\cline{2-4}
\partialinput{10}{24}{../Common/content/tables/table_Dif_Mom_Intelligence_Dif_Child_AFI_9.tex}
\end{longtable}\pagebreak

\begin{longtable}{@{\extracolsep{5pt}}lccc} 
\caption{Gen2 Intelligence Differences $\rightarrow$ Gen2 AFI Differences}\label{table_Dif_Child_Intelligence_Dif_Child_AFI_9}
\\[-1.8ex]\hline 
\hline \\[-1.8ex] 
& \multicolumn{3}{c}{\textit{Dependent variable:} Differences in Gen2 Mean AFI} \\ 
\cline{2-4}
\partialinput{10}{24}{../Common/content/tables/table_Dif_Child_Intelligence_Dif_Child_AFI_9.tex}
\end{longtable}\pagebreak

\begin{longtable}{@{\extracolsep{5pt}}lccc} 
\caption{Dif Joint Intelligence $\rightarrow$ Gen2 Dif AFI}\label{table_Dif_Joint_Intelligence_Dif_Child_AFI_9}
\\[-1.8ex]\hline 
\hline \\[-1.8ex] 
& \multicolumn{3}{c}{\textit{Dependent variable:} Differences in Gen2 Mean AFI} \\ 
\cline{2-4}
\partialinput{10}{26}{../Common/content/tables/table_Dif_Joint_Intelligence_Dif_Child_AFI_9.tex}
\end{longtable}\end{landscape}