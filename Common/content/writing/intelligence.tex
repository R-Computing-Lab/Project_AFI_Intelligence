\subsubsection{Generation 1 Intelligence}
The Armed Services Vocational Aptitude Battery (ASVAB; Form 8A; \citealp{Palmer1988}) was administered to Gen1 participants in 1980. The Armed Forces Qualification Test (AFQT) is contained within the ASVAB, and has been used in the US military as a measure of general trainability \citep{maier1986asvab}. It is a composite of four subscales: Arithmetic Reasoning (AR; 30 items), Math Knowledge (MK; 25 items), Paragraph Comprehension (PC; 15 items), and Word Knowledge (WK; 35 items). Other administrations of the pencil and paper ASVAB reveal that all the AFQT subscales have high coefficient $\alpha$ internal consistency ( $\alpha_{AR} = .91$; $\alpha_{WK} = .92$; $\alpha_{PC} = .81$; $\alpha_{MK} = .87$; \citealp{kass1982}). Reported reliability of the overall AFQT (version 8A) ranges from .87 to .93 \citep{Palmer1988}.

Methods of calculating the AFQT have varied throughout the ASVAB's administrative lifetime \citep{mayberry1992computing}. For pencil and paper administrations, standard scores were created for each of the subscale scores ($\bar{x}=50$, sd = 10), and then combined into a standard score. Then, the AFQT standard score is derived from the following formula:\begin{align}\text{AFQT} = \text{AR} + \text{MK} + 2\text{VE}, \\\text{where, VE} = \text{PC} + \text{WK.}\end{align}

Many researchers have used the AFQT standard score as a proxy for general intelligence (\textit{g}) \citep{herrnstein1994bell,Der2009}. Indeed, the U.S. military has found that the AFQT correlated 0.8 with the Wechsler Adult Intelligence Scale (WAIS; \citealp{mcgrevy1974relationships}). Moreover, the AFQT consistently predicts outcomes traditionally associated with intelligence\citep{Welsh1990}, including grades \citep{wilbourn1984,mathews1977analysis}.

\subsubsection{Generation 2 Intelligence}
NLSY-Children respondents, beginning at age five and as a consistent part of the survey, complete the following test batteries: \begin{itemize}
\item Peabody Individual Achievement Test (PIAT; \citealp{dunn1970peabody}):\begin{itemize}\item Math Subtest (84 items),
\item Reading Recognition Subtest (84 items),
\item Reading Comprehension Subtest (84 items),\end{itemize}
\item The Peabody Picture Vocabulary Test-Revised (PPVT-R; Form L; \citealp{dunn1981peabody}; 175 items), and
\item Digit Span Subscale of the Wechsler Intelligence Scales for Children--Revised (Digit Span; \citealp{wechsler1974manual}; 28 items).\end{itemize}
The standard scores of the PPVT-R, PIATs, and Digit Span are considered valid and reliable assessments of cognitive ability \citep{mott1995nlsy}. However, subjects were surveyed on a biennial basis. Thus we could not use cognitive tests at a fixed age. Instead, we aggregated scores across a 4 year window, and targeted the midpoint between ages 9 and 10. We targeted 9.5 because all cognitive tests were administered within the 8--11 age window, we wanted to maximize the number of subjects with viable ability scores, and we wanted to ensure temporal precedence by measuring intelligence prior to the occurrence of AFI. In the case of missing subtests, we allowed age 11 standard scores to replace age 9 standard scores, and age 8 standard scores to replace age 10 standard scores. Our replacement strategy ensured that the average age of testing matched the average of our targeted ages. To obtain intellectual ability measures for each NLSY-children respondent, we fit a confirmatory factor analysis model \citep[using Mplus;][]{mplus} and their robust maximum likelihood estimator option. A single-factor model fit moderately well (RMSEA = .101; CFI = .973; TLI = .946), and we used this model to construct a unidimensional scale score for each respondent. We used factor scores obtained from this model as our measure of NLSY-Children intelligence.

\subsubsection{Replicability and Reliability} We repeated our aggregates of Gen2 intelligence, centered at ages 10.5 and 11.5, and replicated all of our analyses. These replications can be found in the Appendices \ref{appen10} and \ref{appen11}, respectively. The test-retest reliabilities of Gen2 intelligence across our three aggregations is reported in the lower triangle of Table \ref{table_measurement_trt_g2int}. The diagonal indicates the number of respondents with intelligence aggregations for that year, and upper triangle reveal the number of respondents with viable scores for both respective ages. The test-retest correlations are very high (r > .90) across all pairings, suggesting that our method captures consistent measures of intelligence across ages.\medskip\\
\noindent\begin{minipage}{\linewidth}
\begin{longtable}{@{\extracolsep{5pt}}rlll} \caption{\small Gen2 Aggregated Intelligence Correlations and Sample Sizes (Ages 9.5, 10.5, 11.5)}\label{table_measurement_trt_g2int}
\partialinput{6}{12}{../Common/content/tables/table_ttintreliable_z.tex}
\end{longtable}
\end{minipage}
