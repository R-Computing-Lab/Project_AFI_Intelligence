We adapted Kenny and colleagues \citeyearpar{kenny2001social,kenny2006dyadic} reciprocal standard dyad model to facilitate sibling comparisons. Sibling-based quasi-experimental models are particularly effective for incorporating genetic and environmental design elements \citep{Lahey2010,Rutter2007}. Our model uses differences between both pairs of mothers and pairs of adolescents, which provides the measures that explicitly account for within-family variance. Further, within-family differences create a powerful control for virtually all background heterogeneity (variance) associated with both genetic and environmental differences (\citeauthor{Lahey2010}). We compare individuals from within the family in the context of the following models. First, we predict the difference in AFI, $Y_{i\Delta}$, for a given pair of NLSY-Children, indexed as i, in the following model:
\begin{equation}\label{equation_discord_main}
\mathrm{Y_{i\Delta}} = \beta_0 + \beta_1\mathrm{\bar{Y_{i}}} + \beta_2\mathrm{\bar{X_{i}}} + \beta_3\mathrm{X_{i\Delta}}
\end{equation}\vspace{-10pt}
where,
\begin{equation}\label{equation_discord_defs_delta}
\mathrm{Y_{i\Delta}} = \mathrm{Y_{i1}} - \mathrm{Y_{i2}};\, \mathrm{X_{i\Delta}} = \mathrm{X_{i1}} - \mathrm{X_{i2}},\, \mathrm{and}
\end{equation}
\begin{equation}\label{equation_discord_defs_min}
\mathrm{Y_{i1}} = \mathrm{max}(\mathrm{Y_{ij}});\, \mathrm{Y_{i2}} = \mathrm{min}(\mathrm{Y_{ij}})
\end{equation}\\

In this model, the relative difference in kin outcomes ($\mathrm{Y_{\Delta}}$ \eg, AFI) is predicted from the mean level of Y ($\mathrm{\bar{Y}}$, \eg mean AFI), the mean level of X ($\mathrm{\bar{X}}$ \eg, intelligence), and the between-kin intelligence difference ($\mathrm{X_{i\Delta}}$). The mean levels support causal inference through at least partial control for genes and shared environment in previous generations. Within this model, there is explicit separation of within-family variance (within $\mathrm{Y_{\Delta}}$ and $\mathrm{X_{\Delta}}$), and between-family variance (within $\mathrm{\bar{Y}}$ and $\mathrm{\bar{X}}$).

Thus, this model allows us to explicitly untangle between- and within-family influences. If there is a true causal link between intelligence and AFI, then we expect kin differences in intelligence to be significantly associated with kin differences in AFI. If the effect is spurious -- only a function of between-family confounds -- then we would expect to find no significant relationship between the differences in the outcome with the differences in the predictor.